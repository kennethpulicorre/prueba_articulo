\begin{abstract}
La virtualización basada en contenedores ha logrado consolidarse en los entornos académicos y de la industria gracias al rendimiento y su alta densidad de ambientes virtuales en un ambiente anfitrión. 
El ecosistema de contenedores y su bajo \emph{footprint} han facilitado la distribución de ambientes virtuales de contenedores favoreciendo la reproducibilidad no solo de entornos productivos sino también de ambientes de pruebas.
Gracias a estos factores hay cada vez más entornos que adoptan estas tecnologías en sus líneas de producción y gestión de información. 
Sin embargo, los cambios y las mejoras se hacen necesarios.
Se identifican posibles vulnerabilades y oportunidades de mejoran que favorecen la adopción de cambias y nuevas características y la aparición de nuevas alternativas de virtualización en este contexto. 
Así mismo, el kernel del sistema operativo se modifica y esto afecta el rendimiento de los entornos virtualizados.
Este artículo busca consolidar la evaluación de tres de las herramientas de virtualización más populares como son: LXC, Docker y CoreOS. 
Estas herramientas son puestas a prueba a la hora de acceder a la CPU, memoria RAM, disco y red. 
Así mismo, se evalúa el rendimiento de un par de motores de bases de datos, uno en el área de las bases de datos relacionales y  otro de las bases de datos NoSQL, sobre los tres entornos de virtualización descritos anteriormente.
\end{abstract}
